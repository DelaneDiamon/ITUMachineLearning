\documentclass{article}
\title{Decision Trees Homework}
\author{Dmitrii Dunin, ITU ID 94739}
%\DeclareGraphicsExtensions{.png,.pdf}
\usepackage{Sweave}
\begin{document}
\input{Dmitrii-concordance}
\begin{center}
{\bf\Large Decision Trees Homework}
\end{center}
\begin{center}
{\Large Dmitrii Dunin, ITU ID 94739}
\end{center}
\begin{center}
{\bf\Large International Technological University}
\end{center}

\section*{Question 1}

\begin{Schunk}
\begin{Sinput}
> rm(list=ls())
> require(graphics)
> require(stringr)
> setwd("F:/Workspace/R/Homework2")
\end{Sinput}
\end{Schunk}

1) This question uses the following ages for a set of trees: 19, 23, 30, 30, 45,
25, 24, 20. Store them in R using the syntax ages<-c(19, 23, 30, 30, 45, 25, 24, 20).
\begin{Schunk}
\begin{Sinput}
> ages<-c(19, 23, 30, 30, 45, 25, 24, 20)
\end{Sinput}
\end{Schunk}
\subsection*{1a)}
a) Compute the standard deviation in R using the sd() function. Also compute the mean and median.
\begin{Schunk}
\begin{Sinput}
> sd(ages)
\end{Sinput}
\begin{Soutput}
[1] 8.315218
\end{Soutput}
\begin{Sinput}
> mean_ages<-mean(ages)
> mean_ages
\end{Sinput}
\begin{Soutput}
[1] 27
\end{Soutput}
\begin{Sinput}
> median_ages<-median(ages)
> median_ages
\end{Sinput}
\begin{Soutput}
[1] 24.5
\end{Soutput}
\end{Schunk}
\subsection*{1b)}
b) Compute the same value in R without the sd function.
\begin{Schunk}
\begin{Sinput}
> sqrt(sum((ages-mean(ages))^2/(length(ages)-1)))
\end{Sinput}
\begin{Soutput}
[1] 8.315218
\end{Soutput}
\end{Schunk}
\subsection*{1c)}
c) Using R, how does the standard deviation from part a) change if you add
10 to all the values?
\begin{Schunk}
\begin{Sinput}
> ages_plus_10<-ages+10
> sd(ages_plus_10)
\end{Sinput}
\begin{Soutput}
[1] 8.315218
\end{Soutput}
\end{Schunk}
Standard deviation does not change
\subsection*{1d)}
d) Using R, how does the standard deviation in part a) change if you
multiply all the values by 100?
\begin{Schunk}
\begin{Sinput}
> ages_multiply_100<-ages*100
> sd(ages_multiply_100)
\end{Sinput}
\begin{Soutput}
[1] 831.5218
\end{Soutput}
\end{Schunk}
Standard deviation also gets multiplied by 100
\subsection*{1e)}
e) Next add another tree of age 70 to the sample. Compute the mean and
median with this tree added to the sample. How have the mean and
median changed?
\begin{Schunk}
\begin{Sinput}
> ages_add_70<-append(ages, 70)
> mean_ages_70<-mean(ages_add_70)
> mean_ages_70
\end{Sinput}
\begin{Soutput}
[1] 31.77778
\end{Soutput}
\begin{Sinput}
> mean_ages_70 - mean_ages
\end{Sinput}
\begin{Soutput}
[1] 4.777778
\end{Soutput}
\begin{Sinput}
> median_ages_70<-median(ages_add_70)
> median_ages_70
\end{Sinput}
\begin{Soutput}
[1] 25
\end{Soutput}
\begin{Sinput}
> median_ages_70 - median_ages
\end{Sinput}
\begin{Soutput}
[1] 0.5
\end{Soutput}
\end{Schunk}

\section*{Question 2}

\begin{Schunk}
\begin{Sinput}
> rm(list=ls())
\end{Sinput}
\end{Schunk}

2) Here is the data table for question 2.
\begin{Schunk}
\begin{Sinput}
> TrainData<-read.csv("HW02DataTrain.csv",header=TRUE)
> str(TrainData)
\end{Sinput}
\begin{Soutput}
'data.frame':	9 obs. of  4 variables:
 $ a1    : logi  TRUE TRUE TRUE FALSE FALSE FALSE ...
 $ a2    : logi  TRUE TRUE FALSE FALSE TRUE TRUE ...
 $ a3    : int  1 6 5 4 7 3 8 7 5
 $ Target: Factor w/ 2 levels "-","+": 2 2 1 2 1 1 1 2 1
\end{Soutput}
\begin{Sinput}
> summary(TrainData)
\end{Sinput}
\begin{Soutput}
     a1              a2                a3        Target
 Mode :logical   Mode :logical   Min.   :1.000   -:5   
 FALSE:5         FALSE:4         1st Qu.:4.000   +:4   
 TRUE :4         TRUE :5         Median :5.000         
 NA's :0         NA's :0         Mean   :5.111         
                                 3rd Qu.:7.000         
                                 Max.   :8.000         
\end{Soutput}
\end{Schunk}
\subsection*{2a)}
The following tree was created using rpart for the data table given above.
\begin{Schunk}
\begin{Sinput}
> fit<-rpart(Target~a1+a2+a3, 
+       data=TrainData, 
+       method="class",
+       control=rpart.control(minsplit=0,minbucket=0,maxdepth=5))
> fit
\end{Sinput}
\begin{Soutput}
n= 9 

node), split, n, loss, yval, (yprob)
      * denotes terminal node

 1) root 9 4 - (0.5555556 0.4444444)  
   2) a1< 0.5 5 1 - (0.8000000 0.2000000)  
     4) a2>=0.5 3 0 - (1.0000000 0.0000000) *
     5) a2< 0.5 2 1 - (0.5000000 0.5000000)  
      10) a3>=6 1 0 - (1.0000000 0.0000000) *
      11) a3< 6 1 0 + (0.0000000 1.0000000) *
   3) a1>=0.5 4 1 + (0.2500000 0.7500000)  
     6) a2< 0.5 2 1 - (0.5000000 0.5000000)  
      12) a3< 6 1 0 - (1.0000000 0.0000000) *
      13) a3>=6 1 0 + (0.0000000 1.0000000) *
     7) a2>=0.5 2 0 + (0.0000000 1.0000000) *
\end{Soutput}
\end{Schunk}
Use this tree to predict the class labels (either a + or -) for the following test
observations:
\begin{Schunk}
\begin{Sinput}
> PredictData<-read.csv("HW02DataPredict.csv",header=TRUE)
> str(PredictData)
\end{Sinput}
\begin{Soutput}
'data.frame':	4 obs. of  3 variables:
 $ a1: logi  TRUE TRUE FALSE FALSE
 $ a2: logi  TRUE FALSE TRUE FALSE
 $ a3: num  2.5 5.5 2.5 8.5
\end{Soutput}
\begin{Sinput}
> predict(fit, PredictData, type="class")
\end{Sinput}
\begin{Soutput}
1 2 3 4 
+ - - - 
Levels: - +
\end{Soutput}
\end{Schunk}
\end{document}
